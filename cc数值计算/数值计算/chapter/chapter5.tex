\section{第五章}



\subsection{题目1}

自行编写复合梯形公式、Simpson公式的计算程序 \footnote{注:本题为验证公式的正确性,选用《数值方法》课本例题作为验证,验证程序的正确性。该题目的要求为,$y=2+sin(2\sqrt{x})$,设立10个区间(即11个点),计算从1到6的积分值。}


\paragraph{组合梯形公式}

设函数划分宽度为$h$,一共有$M+1$个点,即$M$个区间。考虑到计算的简便性,采用以下公式计算函数的积分值
$$T(f,h)=\frac{h}{2}(f(a)+f(b))+h\sum_{k=1}^{M-1} f(x_k)$$

经过计算,对$y=2+sin(2\sqrt{x})$的积分值为:8.193854565172531

\paragraph{Simpson公式}

设一共有2M个区间,每个区间等距,距离为$h$,选用组合辛普森公式公式一进行计算,即
$$S(f,h)=\frac{h}{3}\sum_{k=1}^{M}(f(x_{2k-2})+4f(x_{2k-1})+f(x_{2k}))$$

经过计算,对$y=2+sin(2\sqrt{x})$的积分值为:8.183472811131802

\paragraph{代码}
\begin{minted}{python}
def trapezoid(f, a, b, M):
    sum = 0
    h = (b-a)/M
    # print(h)
    for i in range(1, M):
        sum += f(i*h+a)
        # print(f(i*h+a))

    return (h/2)*(f(a)+f(b))+sum*h
    
    
def simpson(f, a, b, M):
    h = (b - a) / (2 * M)
    s = 0
    for k in range(1, M + 1):
        s += f(a + (2 * k - 2) * h) + 4 * f(a + (2 * k - 1) * h) + f(a + 2 * k * h)
    s *= h / 3
    return s


\end{minted}

\subsection{题目2}

取$h=0.01$,分别利用复合梯形公式和辛普森公式,计算定积分
$$I(x)=\frac{1}{\sqrt{2 \pi}} \int_{0}^{1} \exp ^{-\frac{x^{2}}{2}} \mathrm{d} x$$
试与精确解比较,说明两种格式的优劣。

\paragraph{解答}
通过对\textit{题目一}函数的调用,分别求得利用复合梯形公式和辛普森公式的积分解为:
\[
0.34134272963911727\]
\[
0.34134474607022336
\]

查表得,原函数的积分值为$\phi(1)-\phi(2)$,计算得
原函数积分的值为:
\[
	0.3413447460685431
\]
可知,辛普森公式的计算精度更高


\paragraph{代码}


\begin{minted}{python}


def trapezoid(f, a, b, M):
    sum = 0
    h = (b-a)/M
    # print(h)
    for i in range(1, M):
        sum += f(i*h+a)
        # print(f(i*h+a))

    return (h/2)*(f(a)+f(b))+sum*h

def simpson(f, a, b, M):
    h = (b - a) / (2 * M)
    s = 0
    for k in range(1, M + 1):
        s += f(a + (2 * k - 2) * h) + 4 * f(a + (2 * k - 1) * h) + f(a + 2 * k * h)
    s *= h / 3
    return s


def I_simpson(h):
    M=int(2/h)
    c=1/(math.sqrt(2*math.pi))
    print(c*simpson(g,0,1,M))
    
def I_trapezoid(h):
    M=int(1/h)
    c=1/(math.sqrt(2*math.pi))
    print(c*simpson(g,0,1,M))
    
I_simpson(0.01)
I_trapezoid(0.01)
\end{minted}






\subsection{题目3}
若取计算精度为$10^{-4}$,则h=?,n=?

\subsubsection{Simpson公式求解}
取h=0.5,n=2(原公式为M=1)时,求得的积分解为0.3415290519962957
取h=0.25,n=4(原公式为M=2)时,求得的积分为0.34135548785664915,误差$e=1.074178810606119e-05$小于$10^{-4}$。故精度为$10^{-4}$时,h=0.25,n=4;



\subsubsection{复合梯形公式求解}
取$h_0=0.01$,并依次增加,不断求出复合梯形公式的计算结果与精确解之间的误差,前10次迭代结果为:

$$h_{0}=0.01,I(x)=-0.000002016$$
$$h_{1}=0.02,I(x)=-0.000008066$$
$$h_{2}=0.03,I(x)=-0.000018517$$
$$h_{3}=0.04,I(x)=-0.000032265$$
$$h_{4}=0.05,I(x)=-0.000050415$$
$$h_{5}=0.06,I(x)=-0.000078777$$
$$h_{6}=0.07,I(x)=-0.000102896$$
$$h_{7}=0.08,I(x)=-0.000140062$$
$$h_{8}=0.09,I(x)=-0.000166692$$
$$h_{9}=0.10,I(x)=-0.000201710$$
$$h_{10}=0.11,I(x)=-0.000249044$$


即,当精度为$10^-4$时,h最大为0.06,但由于n不是整数,故h应取0.05,n应取20;










