\section{总结}

计算机科学根本上是一门抽象的科学,算法是其灵魂,数学是其基础,要求很高的实践性,而《数值计算》这门课几乎涵盖了计算机科学的这几大部分,既有适用于计算机求解的实用数值算法,又有严谨的数学推导与证明,还为许多高级算法、应用程序提供了基础设施的支持,有着很强的实践性,在许多任务中,这些基本的数值方法往往是效率的关键。

《数值计算》这门课在我的整个本科阶段的学习中也起到了承前启后的作用,既承接了之前的《高等数学》、《线性代数》等数学基础课程,又为后面更加高级的《算法分析》、《图形学》和《机器学习》等课程奠定了基础。在这门课中,我进一步体会到了数学与算法之美,能够更加注重自己抽象思维和计算思维的培养,更加注重算法功底的牢固,更加注重数学知识的深入与拓宽,更加注重实践,通过这几次实验,能够在实践中深化和检验理论。

感谢刘保东老师的辛勤付出!