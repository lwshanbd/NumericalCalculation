\section{作业}


\subsection{1.1}

完成下列计算:
$\int_{0}^{1 / 4} e^{x^{2}} d x \approx \int_{0}^{1 / 4}\left(1+x^{2}+\frac{x^{2}}{2 !}+\frac{x^{6}}{3 !}\right) d x=\widehat{p}$
指出在这种情况下会出现哪种类型的误差,并将计算结果与真实值p=0.2553074606进行比较

\textbf{解答}

\begin{equation}
\begin{aligned} \int_{0}^{1 / 4} e^{x^{2}} d x & \approx \int_{0}^{1 / 4}\left(1+x^{2}+\frac{x^{4}}{3}+\frac{x^{6}}{3 !}\right) d x \\ &=\left(x+\frac{x^{3}}{3}+\frac{x^{5}}{5(2 !)}+\frac{x^{7}}{7(3 !)}\right)^{x=\frac{1}{4}}_{x=0}\\ &=\frac{1}{4}+\frac{1}{\sqrt{9}^{2}}+\frac{1}{10240}+\frac{x}{7(381)} \\ &=\frac{192807^{2}}{1146880} \approx 0.2553074428=\hat{p} \end{aligned}
\end{equation}



会出现截断误差,这导致了最后解的值小于真实值。

\subsection{1.2}

有时使用三角或代数恒等式,重新排列函数中的项,可以避免精度损失。求下列函数的等价公式,以避免精度损失。

1. $ln(x+1)-ln(x)$,其中x较大

2.$\sqrt{x^{2}+1}-x$,其中x较大

3.$\cos ^{2}(x)-\sin ^{2}(x)$,其中x约等于$\pi/4$

4.$\sqrt{\frac{1+\cos (x)}{2}}$,其中x约等于$\pi$



\textbf{解答}

1. 将原式$ln(x+1)-ln(x)$转换为$\ln \left(\frac{x+1}{x}\right)$

2.将原式$\sqrt{x^{2}+1}-x$转换为$\frac{1}{\sqrt{x^{2}+1+x}}$

3.将原式$\cos ^{2}(x)-\sin ^{2}(x)$转换为$\cos (2 x)$.

4.将原式$\sqrt{\frac{1+\cos (x)}{2}}$转换为$\cos (x / 2)$


\subsection{1.3}


讨论下列计算过程中的误差传播

(1)三个数的和:$$
p+q+r=\left(\widehat{p}+\epsilon_{p}\right)+\left(\widehat{q}+\epsilon_{q}\right)+\left(\widehat{r}+\epsilon_{r}\right)
$$

(2)两个数的商:$$\frac{p}{q}=\frac{\hat{p}+\epsilon_{p}}{\hat{q}+\epsilon_{q}}$$

(3)三个数的积:$$p q r=\left(\hat{p}+\epsilon_{p}\right)\left(\hat{q}+\epsilon_{q}\right)\left(\hat{r}+\epsilon_{r}\right)$$

\textbf{解答}

1.\begin{equation}\epsilon_{p}+\epsilon_{q}+\epsilon_{r}
\end{equation}
2.\begin{equation}
\frac{p}{q}=\frac{\hat{p}+\epsilon_{p}}{\hat{q}+\epsilon_{q}}=\frac{\hat{p}}{\hat{q}}+\frac{\epsilon_{p}+\frac{p}{\hat{q}} \epsilon_{q}}{\hat{q}+\epsilon_{q}}
\end{equation}


3.
\begin{equation}
\begin{aligned} p q r=&\left(\hat{p}+\epsilon_{p}\right)\left(\hat{q}+\epsilon_{q}\right)\left(\hat{r}+\epsilon_{r}\right) \\=& \hat{p} \hat{q} \hat{r}+\hat{p} \hat{r} \epsilon_{q}+\hat{q} \hat{r} \epsilon_{p}+\hat{p} \hat{q} \epsilon_{r}+\hat{r} \epsilon_{p} \epsilon_{q}+\hat{q} \epsilon_{p} \epsilon_{r}+\hat{p} \epsilon_{q} \epsilon_{r}+\epsilon_{p} \epsilon_{q} \epsilon_{r} \\=& \hat{p} \hat{q} \hat{r}+\left(\hat{p} \hat{r} \epsilon_{q}+\hat{q} \hat{r} \epsilon_{p}+\hat{p} \hat{q} \epsilon_{r}\right) \\ &+\left(\hat{r} \epsilon_{p} \epsilon_{q}+\hat{q} \epsilon_{p} \epsilon_{r}+\hat{p} \epsilon_{q} \epsilon_{r}\right)+\epsilon_{p} \epsilon_{q} \epsilon_{r} \end{aligned}
\end{equation}


\subsection{1.3}

设有泰勒展开式

$$\cos (h)=1-\frac{h^{2}}{2 !}+\frac{h^{4}}{4 !}+O\left(h^{6}\right)$$

和


$$\sin (h)=h-\frac{h^{3}}{3 !}+\frac{h^{5}}{5 !}+O\left(h^{7}\right)$$

判定它们和与积的近似阶:

\textbf{解答}

\begin{equation}
\begin{aligned} \cos (h)+\sin (h) &=1+h-\frac{h^{2}}{2}-\frac{h^{3}}{6}+\frac{h^{4}}{24}+\boldsymbol{O}\left(h^{5}\right) \\ \cos (h) \sin (h) &=h-\frac{2 h^{3}}{3}+\frac{2 h^{5}}{15}+\boldsymbol{O}\left(h^{7}\right) \end{aligned}
\end{equation}

\begin{equation}
\begin{array}{l}{\text { 中间的计算过程为}} \\ {\qquad\left(1+h+\frac{h^{2}}{2 !}+\frac{h^{3}}{3 !}+\frac{h^{4}}{4 !}\right)\left(h-\frac{h^{3}}{3 !}\right)=h+h^{2}+\frac{h^{3}}{3}-\frac{h^{5}}{24}-\frac{h^{6}}{36}-\frac{h^{7}}{144}}\end{array}
\end{equation}




\subsection{分析讨论题1}

求方程$x^2+(\alpha+\beta)x+10^9=0$的根,其中$\alpha=-10^9,\beta=-1$讨论如何设计计算格式才能有效地减少误差,提高计算精度。

\paragraph{分析}


设$a \neq 0, b^2 - 4ac > 0$,且有方程$ax^2 + bx + c = 0$,则通过如下二次根公式可解出方程的根:

\begin{equation}
x_1=\frac{-b+\sqrt{b^2-4ac}}{2a}  \quad \quad x_2=\frac{-b-\sqrt{b^2-4ac}}{2a}
\label{eq3}
\tag{1}
\end{equation}

通过将分子有理化,可以等价变换成下列公式

\begin{equation}
x_1=\frac{-2c}{b+\sqrt{b^2-4ac}} \quad \quad x_2=\frac{-2c}{b-\sqrt{b^2-4ac}}
\label{eq4}
\tag{2}
\end{equation}

\textbf{当$|b| \approx \sqrt{b^2 - 4ac}$,必须小心处理,以避免其值过小而引起巨量消失(catastrophic cancellation)而带来精度损失。}

\begin{itemize}
	\item 当$b > 0$的时候应使用公式(\ref{eq4})计算$x_1$,应使用公式(\ref{eq3})计算$x_2$。
	\item 当$b < 0$的时候应使用公式(\ref{eq3})计算$x_1$,应使用公式(\ref{eq4})计算$x_2$。
\end{itemize}



综上,应使用公式(\ref{eq3})计算$x_1$,应使用公式(\ref{eq4})计算$x_2$。



\subsection{分析讨论题2}

以计算$x^31$为例,讨论如何设计计算格式才能减少计算次数

\paragraph{解答}



