\section{综述}

数值计算指有效使用数字计算机求数学问题近似解的方法与过程, 主要研究如何利用计算机更好的解决各种数学问题, 包括连续系统离散化和离散形方程的求解, 并考虑误差、收敛性和稳定性等问题.

从数学类型来分, 数值运算的研究领域包括数值逼近、数值微分和数值积分、数值代数、最优化方法、常微分方程数值解法、积分方程数值解法、偏微分方程数值解法、计算几何、计算概率统计等. 随着计算机的广泛应用和发展, 许多计算领域的问题, 如计算物理、计算力学、计算化学、计算经济学等都可归结为数值计算问题.

本学期实验涉及到了非线性方程的求解、线性方程组的数值解、插值与多项式逼近、曲线拟合、数值微积分、微分方程求解等算法。笔者使用Python语言编程实现,每个算法都给出了详尽的说明、必要的证明和精美的图表。

其中,Python版本为3.7,涉及到的Python库包括但不限于:Scikit-Learn, Scipy,  pandas,numpy.上述Python库均为开源库,同时,本文使用\LaTeX 编写,\LaTeX 源码和Python源码已经在GitHub开源\footnote{https://github.com/ethereum/go-ethereum}。